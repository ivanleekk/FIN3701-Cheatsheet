\section{Dividends}
Dividend Discount Model: $P_0 = \frac{D_1}{r-g}$, where $D_1$ is the dividend in the \textbf{next period}, $r$ is the required rate of return, and $g$ is the growth rate of the dividend.

FCFs are either retained as earnings to invest in new projects or as reserves, or to be distributed to shareholders as dividends or share repurchases.

Dividends are not a legal right, but most will pay regular dividends at a mostly constant level, or a large special dividend when there is extra cash in the company. In the most extreme case, a liquidating dividend is a return of capital to shareholders when the firm is terminated.

Dividend Timeline:
\begin{itemize}
    \item Declaration Date: Board declares the dividend
    \item Cum-Dividend Date: Shareholders who buy shares before or on this date will receive the dividend
    \item Ex-Dividend Date: Shareholders who buy shares after this date will not receive the dividend
    \item Record Date: Shareholders who own shares on this date will receive the dividend
    \item Payment Date: Dividend is paid to shareholders
\end{itemize}

In a perfect world, the stock price will fall by the dividend on the ex-dividend date.

\textbf{Homemade Dividends}:
If a firm does not pay dividends, shareholders can create their own dividends by selling off some of their shares. This is called a home-made dividend.

\textbf{Stock Repurchases}:
Firms can also buy back their own shares to return cash to shareholders. This is called a stock repurchase.

Either via Open Market, or Tender Offer at Fixed price.

\textbf{Dividend Irrelevance Theory}:
The theory that dividends do not affect the value of the firm, as shareholders can create their own dividends by selling off shares if they need cash. And the value of this will be equal to if the firm had paid out that amount as a dividend in the first place.

There will be no change in value unless the firm's investment policy changes.

$\therefore$ Dividends are irrelevant to the value of the firm.
Thus firms should also never forgo a positive NPV project just to pay out dividends.

In a world with taxes, dividends are also taxed (Not in Singapore), so there is also a tax disadvantage to receiving dividends. If the capital gains tax is lower than the dividend tax, then it is preferable to not pay dividends so that the taxes can be deferred.

In places where dividends are taxed, firms may prefer to buy back shares instead of paying dividends. As this can generate more value for the shareholders.

For share repurchases, firm's should buy back before good news is released, as the stock price will rise after the good news is released. This is called the signalling effect. Or only after bad news is released, to make it cheaper to buy back shares.

\textbf{Clientele Effect}:
The idea that different investors have different preferences for dividends, and firms will attract different investors based on their dividend policy.

These investors may like to have regular income from dividends and thus this may attract them to invest in a dividend paying firm over an identical non-dividend paying firm.

\subsection{Agency Costs}
\textbf{Agency Cost of Equity}:
When firms have excess cash, managers may spend inefficiently for executive perks or over pay for acquisitions. This is an agency cost of equity.

By paying out more excess cash as dividends, there are less resources for managers to waste. Thus maximising stock price.

\textbf{Agency cost between Majority and Minority Shareholders}:
Majority shareholders may pay out dividends to themselves, leaving minority shareholders with less dividends. This is an agency cost between majority and minority shareholders. Since profits might not be allocated on a pro-rated basis.

Majority shareholders with control might prefer retained earnings to control when they have to pay dividend taxes, however interfering with the cash flows for other minority shareholder Since profits might not be allocated on a pro-rated basis.

Majority shareholders with control might prefer retained earnings to control when they have to pay dividend taxes, however interfering with the cash flows for other minority shareholders.

\subsection{Korean Chaebols Case Study}
Chaebols are large Korean conglomerates that are controlled by a single family. They are known for their complex ownership structures and cross-holdings of shares in other companies.

They used to have low payouts (10\%) and dividend yields (1.5\%) with large cash reserves of \$ 70 Billion USD for Samsung and \$ 22 Billion USD for Hyundai.

Investors didn't use to care about dividends as they were more interested in the growth of the companies for capital gains.


