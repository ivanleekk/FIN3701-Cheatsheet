\section{Cases}
\subsection{Grab Listing}
Grab listed via SPAC in 2021, raising \$ 4.5 Billion USD. It was the largest SPAC listing at the time.

Grab was hard to value based on historical financials, as it was a high growth company that was not profitable.

Grab had merged with Uber's Southeast Asian operations in 2018, and had expanded into food delivery, payments, and financial services. But as part of the deal, it had to go public by 2023, less pay a \$2 Billion USD payout to Uber.

Thus SPAC made sense as a fast way to go public, as it would take too long to go through the traditional IPO process.

At a valuation of \$ 40 Billion USD, Grab was the largest SEA company to go public, but may have been overvalued, resulting it its decline in the years since listing.

Back in 2021, it was prime SPAC environment, with many companies going public via SPACs, but the environment has since cooled down. While that allowed Grab to secure favourable terms at the time, it has since cooled down.

SPAC sponsors normally take about 20\% of the equity in the merged entity, leading to a dilution of ownership higher than that of an IPO.

Listing on the NASDAQ also allowed Grab to access a larger pool of investors, but also exposed it to more scrutiny from US investors. And also exposed it to FX risk as its main operations are in SEA which does not use the USD that its stock is traded and raised in.

\subsection{BWI}
Best World International is a maker of skincare products that was listed on the SGX that was in 50 markets.

It is concerned about the disadvatages of being listed on the SGX and is looking for a way to delist.

Due to undervaluation on the SGX, it was eventually delisted by Selective Capital Reduction, where the company bought back shares at a premium to the market price.

By delisting, Best World can eventually increase its valuation without outside influence, and can focus on long-term growth without the pressure of quarterly results.

However, this also means that the company is less liquid, and that shareholders who want to sell after BWI has delisted will have to do so at a discount to the last traded price.

% \subsection{Great Eastern}
% Great Eastern is a life insurance company that is part of the OCBC group. In 2024, OCBC bought over the remaining $\approx$ 10\% shares of Great Eastern that it did not own.
% 
% This was done to simplify the group structure, and to allow OCBC to have full control over Great Eastern's operations.
% 
% However this was done at the lowest Embedded Value ratio multiple for an insurance acquisition in the region, and was seen as a lowball offer by some shareholders. But this was still at a premium to the last traded share price for GEH.
% 
% Given that OCBC already had control over Great Eastern, the acquisition was seen as a formality, and the lowball offer was seen as a way to save costs.
% 
% For the minority shareholders, this was not fair but still reasonable, as they still get a premium over the last traded price, however albeit the stock was still undervalued.
