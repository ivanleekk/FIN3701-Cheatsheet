\section{Capital Structure}
\begin{tabularx}{\linewidth}{>{\raggedright\arraybackslash}p{0.25\linewidth} >{\raggedright\arraybackslash}X} % Adjust the width as needed
    \toprule
    \multicolumn{2}{c}{\textbf{Formulas}}\\
    \midrule
    Firm Value & $V = B + S$, $B$ = Debt, $S$ = Equity \\
    Modigliani-Miller Theorem & $V = V_{\text{Unlevered}} = V_{\text{Levered}}$ \\
    Return on Equity (no taxes) & $ROE = R_{Unlevered} + \frac{B}{S_{Levered}}(R_{Unlevered}-R_{B})$ \\
    Levered Beta & $\beta_{Levered} = \beta_{Unlevered} \times \left(1 + \frac{B}{S}\right)$ \\
    Value of Levered Firm & $V_{Levered} = V_{Unlevered} + PV(\text{Tax Shield}) = V_{Unlevered} + T_C\times B$ \\
    Return on Equity (with taxes) & $ROE = R_{Unlevered} + \frac{B}{S_{Levered}}(R_{Unlevered}-R_{B}) \times (1 - T_C)$ \\
    Miller Model (Incl personal taxes) & $V = V_{\text{Unlevered}} + \left[1-\frac{(1-T_C)\times (1-T_S)}{1-T_B}\right]\times B$ \\
    Trade-Off Model & $V = V_{\text{Unlevered}} + PV(\text{Tax Shield}) - PV(\text{Cost of Financial Distress}) - PV(\text{Agency Cost of Debt} - PV(\text{Agency cost of Equity}))$ \\
    \bottomrule
\end{tabularx}

Capital Structure is how a company does long-term financing, either with \textbf{Equity} or \textbf{Debt}.

Value of a firm is the sum of the value of its debt and equity. The value of the firm should be the same regardless of the capital structure.

Modigliani-Miller Theorem states that the value of a firm is independent of its capital structure. 
They make the following assumptions: Homogeneous Expectations, Homegeneous Business Risk Classes,
Perpetual Cash Flows, Perfecrt Capital Markets, No Taxes, No Bankruptcy Costs, and No Agency Costs.

Homemade Leverage is the idea that investors can replicate the effects of leverage on their own. By borrowing money to buy more shares, 
they can get the same returns as if the company had borrowed money, or vice versa to unleverage.

Since investors can replicate leverage on their own, the value of the firm is independent of its capital structure.

With no taxes, cashflows from debt and equity are the same, so the value of the firm is the same regardless of capital structure.

Levered Beta $B_S$ is the idea that leverage amplifies market risk of a firm's assets

In a MM World with no taxes, the WACC is the same regardless of capital structure.

Tax Shield is the idea that interest payments are tax deductible, so the value of the firm is higher with debt.

Some of the increase in equity risk and return is offset by the interest tax shield.

With taxes but no bankruptcy cost, the WACC of a firm actually decreases as more debt is added due to the effect of a interest tax shield.

Miller Model suggests that we have to look at the net tax benefit of debt,
as equity income is taxed seperately from personal income.

Shareholders in a levered firm receive $(EBIT-r_B\cdot B)(1-T_C)(1-T_S)$, while bondholders receive $r_B\cdot B(1-T_C)$.

If dividends are taxes lower than interest, then the tax advantage of debt for
corporations is reduced.

Financial Distress is when a firm is unable to meet its debt obligations. And under such situations
the debtholders are given rights over assets.

Tyoes of Bankruptcy:
\begin{itemize}
    \item Chapter 7: Liquidation
    \item Chapter 11: Reorganization
\end{itemize}
Costs include, direct costs from legal and administrative fees, indirect costs from loss of customers, suppliers, and employees, and
opportunity costs from lost investments and projects.

These costs make it riskier to leverage a firm, thus lowering the total value of the firm.

Trade-off Model is the idea that there is an optimal capital structure that balances the tax benefits of debt with the costs of financial distress.

Agency Costs of Leverage such as when managers are aligned with shareholders and not bondholders,
there can be incentives to take larger risks, as the downside to shareholders may not go below 0, causing them to 
take negative NPV projects.

Bondholders can try to solve the problem by having negative covenents on their debt to protect 
themselves from the firm taking on too much risk. Positiver covenents can also be used to ensure that the firm
maintains good condition of assets, or does proper audits.

Agency Costs of Equity (Agency benefits of debt) is the idea that debt can help align managers with shareholders, as they have more to lose if the firm goes bankrupt.
Having Leverage reduces the free cash flows of the firm, which reduces managers ability to take
risky projects.

\subsection{Limits to Debt}
Firms at risk of high financial distress use less debt.
Firms with high growth opportunities use less debt, as industries like technology
do not have good collateral for loans.

Factors in Trade-off Model:
\begin{itemize}
    \item Taxes, if corporate tax is higher than bondholder tax, there is an advatnage to debt.
    \item Types of assets, in case of financial distress, does the firm have tangible assets, having them can reduce the cost of taking on debt.
    \item Uncertainty of operating income, if the firm has stable income, it can take on more debt as the risk of it experiencing financial distress is lower.
\end{itemize}

\subsection{Adverse Selection and Asymeetric Information}
Adverse Selection is the idea that firms with high risk are more 
likely to take on debt, as they know they are more likely to go bankrupt.

Asymmetric Information is the idea that managers know more about the 
firm than shareholders, and can take on risky projects that shareholders are unaware of.

The Lemons Principle is the idea that because managers/sellers have more information about
the quality of the firm/product, then their desire to sell it may be seen
as a signal that the product is of low quality, hence deserving of a heavy discount.

\subsection{Pecking Order Theory}
Pecking Order Theory is the idea that firms prefer to use internal financing first, 
then debt, and finally equity.

\subsection{Raising Capital}
For private firms, the initial capital is typically provided by the founders, 
or immediate family, forming the initial equity.

Angel Investors are individuals who provide capital to startups in exchange for 
equity.

Venture Capitalists are firms that provide capital to startups in 
exchange for equity. Typically made up of a general partner who runs the firm and limited 
partners who provide the capital.

Startups issue convertible preferred stocks to the venture capitalists, 
which can be converted to common stocks at a later date to allow the VC to liquidate their position.

Private Equity Firms are firms that invests in private firms and create value
by changing the management, operations, or capital structure of the firm. One way to do so is using Leveraged Buyouts (LBO).
Typically with bigger deals, shorter horizons and lower risks than VC.

Initial Public Offering (IPO) is when a private firm goes public by selling shares 
to the public, enhancing the liquidity of stocks in that firm.

The advantages of IPOs are that it provides liquidity to the founders, and allows the company
greater access to capital through the public markets.

The disadvantages are that it is expensive, and the firm has to disclose more 
information to the public.

Underwriters are investment banks that manage the issuance of new stocks,
they could be either best-effort underwriting, where the underwriter does not 
guarantee the sale of the stocks, or firm commitment underwriting, 
where the underwriter guarantees the sale of the stocks (buying the undersubscribed stocks if there are any).

\begin{callout}
A \textbf{Greenshoe Provision} is a clause in the underwriting agreement that allows the 
underwriter to issue more shares from the issuer at the offering price if the 
demand is high.
\end{callout}

Typically, IPOs are underpriced in order to ensure the average first-day return
is positive, which can be seen as a signal of the firm's quality.

By underpricing the IPO, the firm is able to attract more investors, and
the underwriter can make more money from the commissions as well as not needing to buy
the undersubscribed stocks.

Since the IPO is underpriced, it is expected that the first day of public trading will
see a positive return, which ensures high demand for the stock.

\begin{callout}
    The \textbf{Winner's Curse} is the idea that the winning bidder in an auction is 
    likely to have overpaid for the item, as they are the most optimistic 
    about the value of the item. In this case, buying the IPO at too high of a price and getting all
    the shares you requested for as it was undersubscribed.
\end{callout}

SPAC is a Special Purpose Acquisition Company, which is a shell company that 
goes public to raise capital to acquire a private company, then using a reverse merger, 
make the acquired company public.

SEOs are Seasoned Equity Offerings, which is when a public company issues more shares
to the public. Typically, the stock price drops after an SEO, as the market 
sees it as a signal that the firm is overvalued.

SEOs can be either cash offers where the firm sells new shares to the public, or rights offers
where the firm offers existing shareholders the right to buy new shares at a discount.


