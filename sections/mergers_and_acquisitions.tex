\section{Mergers and Acquisitions}
M\&A to speed up growth, gain market share, or achieve synergy.

When acquired, a company will experience a change on corporate control.

\subsection{Types of M\&A}
\begin{itemize}
    \item Horizontal: Same industry, same stage of production
    \item Vertical: Same industry, different stage of production
    \item Conglomerate: Different industry
\end{itemize}

Synergy is the increase in value due to the merger.
$Synergy = V_{Merged} - V_{Separate}$
It represents the increase in value that the merged company has over the two companies operating separately. Such as revenue enhancement, cost reduction, tax gains and capital requirements.

Diversification is a dubious reason for mergers, as shareholders can diversify their own portfolios at much lower costs than a company.
It only produces gain if it has a cashflow effect and increases debt capacity. Or to increase liquidity of a private firm.

Bidders will make a tender offer for the shares of the target company. The target company will then decide to accept or reject the offer.

They can pay in cash, stock, or a mix of both. The bidder can also use a leveraged buyout, where they use the target's assets as collateral to finance the acquisition.

NPV of an acquisition is the synergy minus the acquisition cost. But in a competitive market, the price will be bid up to the NPV. 

For pure stock offer, price paid to target's owners is $\frac{\text{New shares issued}}{\text{Old shares + New Shares}} \times$ Market value of combined firm

$\therefore NPV = V_{Combined} - V_{Acquirer} - \text{Amount paid to target} = \text{Synergy}-\text{Acquisition Premium}$

Acquirer sometimes do not experience a price increase after announcement due to agency costs, overconfident acquirer and competitive markets.

\subsection{Takeover Defenses}
Target managers frequently oppose takeovers and can do so with the following defenses:
\begin{itemize}
    \item Poison pill: Target shareholders can buy more shares at a discount if a bidder acquires a certain percentage of shares.
    \item Staggered board: Directors are elected at different times, making it harder to replace them.
    \item Golden parachute: Managers get a large payout if they are fired after a takeover.
    \item White knight: Target finds a friendly bidder to avoid the hostile one.
    \item White squire: Target sells a large stake to a friendly investor.
    \item Crown jewel: Target sells its most valuable assets to make itself less attractive.
\end{itemize}
Corporate charter establishes the conditions that allow for a takeover and these can be amended to make acquisitions more difficult. Such as staggering the board or requiring a supermajority for approval of acquisitions.

Divestitures include sales of assets, spin-offs, equity carve-outs, and split-offs. They can be used to raise cash, reduce debt, or focus on core business.

Spinoffs can create value by reducing diversification, reducign empire building and management entrenchment.
This can undo a bad M\&A, improve the focus of the company, and reduce the size of the company. Better capital allocation and better stock performance.
