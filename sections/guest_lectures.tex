\section{CDL Case Study}
Large M\&A deal with Sincere Property Group, a Chinese property developer.

\subsection{Background}
CDL had quite stable high Sales and Earnings per Share (EPS) and Dividends per Share in the years preceding the deal from 2016---2019.

In April 2020, CDL then wanted to take a 51\% stake in Sincere Property Group, a Chinese property developer, for \$1.9 billion.

By October 2020, directors of CDL had started resigning, and in March 2021, Sincere defaulted on its debt.

In May 2021, CDL wrote off its investment to zero. And sold off its stake for \$1 USD in September 2021.

In the years after the deal, CDL's Earnings dropped significantly although its average dividend remained about the same.

\subsection{Analysis}
The deal was a bad one for CDL, as it lost its entire investment in Sincere.

Due dilligence was not done properly, as the deal was done in a rush, and the directors of CDL did not have enough time to properly assess the risks of the deal.

Further exacerbated by the fact that the new CEO Sherman Kwek was appointed recently and may have wanted to stake his place by sealing a major deal. This was an agency cost between management and shareholders.

The financing mix was also a weird one, as instead of using mostly debt and equity, CDL also included Loans, corporate gurantees and liquidity support for Sincere. These additional terms meant that if Sincere defaulted on its debt, CDL would pick up the slack.

There was also additional FX risk as the deal was done mostly in RMB, while CDL's revenue was in SGD. Any change in the FX rate between SGD/RMB would have affected the value of its investment and liabilities.

In the end:
\begin{itemize}
    \item Market Cap loss of \$5.0 Billion SGD
    \item Net Debt increase of \$3.3 Billion SGD
    \item Major stock price drop of > 50\% since end 2019
    \item Removal of stock from the MSCI SG Index
\end{itemize}

\section{Frontier Market Coal Case Study}
Debt and Equity fit into the capital structure of a firm. The optimal capital structure is the one that maximizes the value of the firm.

\subsection{Background}
B rated coal company that borrowed money in 2012 and defaulted in 2015.

Restructured in 2016 and refinanced in 2019.

\subsection{Analysis}
It originally had a low debt to equity ratio, but as it borrowed more money, it increased its leverage.

After the default, its market cap dropped significantly, and it had to restructure its debt into more secured bonds.

However even after paying back its debt to pre-2012 levels, its market cap did not recover to its original levels. As there was a lack in trust in the company's management despite the income being stronger than before.

\subsection{Misc}
To pay in kind is to pay with things other than cash.

Notes when financing, what is the depth of the market (20 Billion in Singapore's Debt market is very different from US market).

Optionality, you can try out accessing financing from multiple players and markets to find the best deal for the firm.

Original Issue Discount (OID) is like a zero coupon bond with a different name.

Cost of Equity in reality is quite hard to find especially for private companies.

CFO job is to reduce cost of capital as it can increase the firm's value by DCF.

Indian Telcos, initially many players with high prices. Then foreign firm wanted to come in and gain market share by undercuttign everyone. One of the other firms decided to play along and take on much debt to survive till the end and allowing it to outlast other competitors, until there are only 2-3 big players.

